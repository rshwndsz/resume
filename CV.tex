% A tabular CV 

% === PREAMBLE ===
% Font size
\documentclass[letterpaper, 10pt, oneside]{article}
\usepackage[utf8]{inputenc}
\usepackage{setspace}
% For better tables
\usepackage{tabularx}
% For multi-page table
\usepackage{longtable}
% For more color names
\usepackage[dvipsnames]{xcolor}
% For Hyperlinks
\usepackage{hyperref}
% For better link colors
\definecolor{dark-purple}{HTML}{00034D}
\hypersetup{colorlinks  = true,
            linkcolor   = blue,
            filecolor   = magenta,
            urlcolor    = dark-purple, 
            pdftitle    = {Russel's CV},
            bookmarks   = true,
            pdfpagemode = FullScreen,
}
% For thicker horizontal rules
\usepackage{booktabs}
% For icons
\usepackage{fontawesome}
% For margins
\usepackage[left=.64in, right=.64in, bottom=.64in, top=.64in]{geometry}
% For the thick underline - https://tex.stackexchange.com/a/443445
\usepackage{lipsum}
\newcommand\dunderline[3][-1pt]{{%
        \setbox0=\hbox{#3}
\ooalign{\copy0\cr\rule[\dimexpr#1-#2\relax]{\wd0}{#2}}}}
% For customization of list environment
\usepackage{enumitem}

% SETUP 
% Page numbers - {arabic}=Arabic numerals, {gobble}=no page numbers, {roman}=Roman numerals
\pagenumbering{arabic}


% === MY COMMANDS ===
% Section title - Uppercase with first letter slightly bigger
\newcommand{\stitle}[1]{\normalsize{\textsc{#1}}}
% Bold & Italic
\newcommand{\bdit}[1]{\textit{\textbf{#1}}}
% Use hyphens instead of bullets for lists
\def\labelitemi{--}
\begin{document}


% === ME ===
\noindent\dunderline[-2ex]{1pt}{\Large{\textbf{Russel Shawn Dsouza} \hspace{0.7\linewidth} }} \\
\normalsize


% === CV ===
\noindent \begin{longtable}{p{0.14\linewidth} p{0.59\linewidth} rp{0.15\linewidth}}
% Address & socials
\stitle{Contact}     & \href{https://nitk.ac.in}{National Institute of Technology Karnataka (NITK)} 
                     & \href{mailto:russel.171ec143@nitk.edu.in}{russel.171ec143@nitk.edu.in} \\

\stitle{Information} & NH66, Srinivasnagar, Surathkal, Mangalore        
                     & \href{https://rshwndsz.github.io}{rshwndsz.github.io} \\
                     & Karnataka, India 575025. 
                     & \href{mailo:rshwndsz@gmail.com}{\faEnvelope}
                       \href{https://github.com/rshwndsz}{\faGithub} 
                       \href{https://linkedin.com/in/rshwndsz}{\faLinkedin}
                       \textcolor{dark-purple}{rshwndsz} \\
\end{longtable}

\vspace{-1.2em}

\noindent \begin{longtable}{@{} p{0.14\linewidth} p{0.8\linewidth}}

% Research interests
\stitle{Research}  & Computer vision, Neuroscience of vision and motor control, Cybernetics, Mixed Reality \\
\stitle{Interests} & \\
\\


% Education
\stitle{Education} & \textbf{National Institute of Technology Karnataka (NIT Karnataka)} \\
                   & Bachelor of Technology, Electronics and Communications Engineering \hfill \hspace{-3em} \textit{Jul 2017\ --\ May 2021} \\
\\


% Publications
% TODO Remove this shitty publication for serious applications
\stitle{Publications} & Lal, S., \textbf{Dsouza, R.}, Maneesh, M., Kanfade, A., Kumar, A., Perayil, G., Alabhya, K., Chanchal, A.K. and Kini, J. \\
                      & \textit{``A Robust Method for Nuclei Segmentation of H\&E Stained Histopathology Images.''} \\
                      & 2020, 7th International Conference on Signal Processing and Integrated Networks (SPIN) (pp.~453--458)\@. IEEE\@.  \\
                      & \textcolor{dark-purple}{DOI\@: 10.1109/SPIN48934.2020.9070874} \\
\\


% Research experience
\stitle{Research}   & \bdit{Winter Research Intern, Deep learning lab, NIT Karnataka} \\
\stitle{Experience} & \bdit{Segmentation of nuclei in histopathology images of kidney, liver and bladder tissues} \\
                    & \textit{Mentored by Dr.\ Shyam Lal} \hfill \hspace{-3em} \textit{Dec 2019\ --\ Feb 2020} \\
                    & \parbox{0.8\textwidth}{%
                        \begin{itemize}[leftmargin=*, itemsep=-0.88ex, topsep=-0.88ex]
                            \item Implemented state of the art models and designed data pipelines for the nuclear segmentation in histopathology images of kidney and liver tissues. 
                            \item Worked on the detection of Urothelial Carcinoma from whole slide images (average dimensions of 80000$\times$50000).
                            \item Built an open-source repository benchmarking segmentation models on histopathology datasets.
                            \item Presented a report reviewing the different methods to perform nuclear segmentation.
                        \end{itemize}
                    }
\\
\\

                    & \bdit{Summer Research Intern, Deep learning lab, NIT Karnataka} \\
                    & \bdit{Segmentation of nuclei in histopathology images of kidney tissues} \\
                    & \textit{Mentored by Dr.\ Shyam Lal} \hfill \hspace{-3em} \textit{May 2019\ --\ Jun 2019} \\
                    & \parbox{0.8\textwidth}{%
                        \begin{itemize}[leftmargin=*, itemsep=-0.88ex, topsep=-0.88ex]
                            \item Designed and debugged efficient implementations of classical image processing algorithms on large datasets.
                            \item Developed and maintained data pipelines for deep learning models.
                            \item Worked on reproducing the results of seminal papers in the field of automated histopathology.
                        \end{itemize}
                    } 
\\
\\

% Work Experience
\stitle{Work}       & \bdit{Frontend Developer and UI Designer} \\
\stitle{Experience} & \bdit{IRIS, NIT Karnakata} \hfill \textit{Aug 2018\ --\ Apr 2019} \\
                    & \parbox{0.8\textwidth}{%
                        \begin{itemize}[leftmargin=*, itemsep=-0.88ex, topsep=-0.88ex]
                            \item Debugged and maintained parts of the frontend code at IRIS --- The official student portal of NIT Karnataka.
                            \item Designed a new UI system from the ground up in Figma.
                            \item Developed the design system in Vue and worked on an integration with the legacy Rails code.
                        \end{itemize}
                    }
\\
\\
                    & \bdit{Python Developer} \\
                    & \bdit{Pinnacle Media, Manipal} \hfill \textit{May 2018\ --\ Jun 2018} \\
                    & \parbox{0.8\textwidth}{%
                        \begin{itemize}[leftmargin=*, itemsep=-0.88ex, topsep=-0.88ex]
                            \item Built and deployed real-time face detection and recognition, using OpenCV, dlib, and scikit-learn, 
                                on a Raspberry Pi as a part of an `employee attendance' system.
                        \end{itemize}
                    }
\\
\\

% Skills
\stitle{Skills} & \\[-2.34ex]
                & \parbox{0.8\textwidth}{%
                        \begin{itemize}[leftmargin=0ex, itemsep=-0.4ex, topsep=-2ex, label={}]
                            \item \bdit{Languages}:               C++, Python, MATLAB, Javascript, C, Verilog, ngSPICE 
                            \item \bdit{Frameworks and packages}: PyTorch, Keras, OpenCV, scikit-learn, Numerical Python 
                            \item \bdit{Web Development}:         React, Express, Node, MongoDB, GraphQL 
                            \item \bdit{Hardware}:                Raspberry Pi, Arduino, Xilinx Artix 7 FPGA 
                            \item \bdit{Natural languages}:       English, Hindi, Kannada 
                        \end{itemize}
                    }
\\
\\


\newpage
% Notable projects
\stitle{Notable}  & \bdit{Satellite detection in images from low-cost telescopes} \hfill \textit{Jul 2020\ --\ Present} \\
\stitle{Projects} & \\[-4ex]
                  & \parbox{0.8\textwidth}{%
                      \begin{itemize}[leftmargin=*, itemsep=-0.88ex, topsep=1.3ex]
                            \item Working on the design and development of a model to detect orbiting objects in the geostationary ring,
                                  from sequences of consecutive frames imaging unknown portions of the sky,
                                  as a part of the `spotGEO' competition by the European Space Agency (ESA).
                        \end{itemize}
                  } \\

                  & \bdit{Identifying Melanoma in images of skin lesions} \hfill \textit{Jun 2020\ --\ Present} \\
                  & \parbox{0.8\textwidth}{%
                        \begin{itemize}[leftmargin=*, itemsep=-0.88ex, topsep=0.2ex]
                            \item Working on building an ensemble network of detection models to accurately detect skin cancer, 
                                  specifically Melanoma, in images of skin lesions 
                                  as a part of the SIIM-ISC Melanoma classification challenge on Kaggle. 
                        \end{itemize}
                    } \\
                    \\[-1.4ex]

                  & \bdit{Image Denoising} \hfill \textit{July 2020} \\
                  & \parbox{0.8\textwidth}{%
                        \begin{itemize}[leftmargin=*, itemsep=-0.88ex, topsep=0.2ex]
                            \item Reproduced a very deep persistent memory network to perform image restoration by removing noise and
                                  predicting uncorrupted images and achieved results comparable to the original paper.
                            \item The model was trained on images from the Berkeley Segmentation Dataset (BSDS300) and 
                                  tested on a modified version of the CIFAR10 dataset.
                        \end{itemize}
                  } \\
                    \\[-1.4ex]

                  & \bdit{Muon Physics} \hfill \textit{Mar 2020} \\
                  & \parbox{0.8\textwidth}{%
                        \begin{itemize}[leftmargin=*, itemsep=-0.88ex, topsep=0.2ex]
                            \item Designed a custom model to classify muon momenta using a tabular dataset of variables and parameters. 
                            \item The model was trained on monte-carlo simulated data from the Cathode Strip Chambers (CSC) 
                                  at the CMS experiment of Large Hadron Collider at CERN.
                        \end{itemize}
                  } \\
                    \\[-1.4ex]

                  & \bdit{Segmentation of brain tumour in MRI images} \hfill \textit{Dec 2019} \\
                  & \parbox{0.8\textwidth}{%
                        \begin{itemize}[leftmargin=*, itemsep=-0.88ex, topsep=0.2ex]
                            \item Reproduced state of the art semantic segmentation models in Keras/TFv1 
                                  to segment brain tumours and surrounding edema from MRI images.
                            \item The model was trained and tested on a part of the Brain Tumour Segmentation (BraTS) dataset.
                        \end{itemize}
                    }  \\
                    \\[-1.4ex]

                  & \bdit{Detecting Ponzi schemes in smart contracts} \hfill \textit{Aug 2019\ --\ Sep 2019} \\
                  & \parbox{0.8\textwidth}{%
                        \begin{itemize}[leftmargin=*, itemsep=-0.88ex, topsep=0.2ex]
                            \item Designed a custom model to detect Ponzi smart contracts deployed on the Ethereum blockchain 
                                  using CNNs and stacked auto-encoders. 
                            \item The model was trained on the raw bytecode of Ethereum smart contracts mined from the Ethereum blockchain 
                                  using Google BigQuery, publicly available Solidity source code of popular smart contracts, 
                                  and a publicly available dataset of known Ponzi schemes.
                            \item Developed in under 48h as a part of a coding sprint.
                        \end{itemize}
                    }  \\
                    \\[-1.4ex]

                  & \bdit{Predicting truth level of news articles} \hfill \textit{Jul 2019\ --\ Aug 2019} \\
                  & \parbox{0.8\textwidth}{%
                        \begin{itemize}[leftmargin=*, itemsep=-0.88ex, topsep=0.2ex]
                            \item Built a model to classify news articles into 6 different categories based on their truth level.
                            \item The model was trained on the LIAR-PLUS dataset containing news articles and
                                  fact-checking justifications from trusted sources.
                        \end{itemize}
                    }  \\
                    \\[-1.4ex]

                  & \bdit{Space Time Adaptive Processing (STAP) Radar} \hfill \textit{Apr 2019} \\
                  & \parbox{0.8\textwidth}{%
                        \begin{itemize}[leftmargin=*, itemsep=-0.88ex, topsep=0.2ex]
                            \item Presented a report on the current state of STAP in Radar Signal Processing. 
                            \item Simulated a radar implementing STAP in Matlab. 
                        \end{itemize}
                    } \\
\\


% Relevant Coursework
\stitle{Relevant}   & Digital Signal Processing, Machine Learning for Neuroimaging \\
\stitle{Coursework} & Digital System Design, Statistical Analysis, Numerical Analysis \\
                    & Embedded System Design, Microprocessors, VLSI Design, Control Systems \\
                    & Data Structures \& Algorithms, Digital Electronics \& Computer Architecture \\
\\


% Achievements, awards & honours
\stitle{Achievements} & Selected as \textbf{full-time research intern} at the ML Lab, RBCCPS, IISc, Bangalore \hfill \textit{Jul 2020} \\ 
                      & to work on ``Simultaneous localization and mapping (SLAM)'' \\
                      & \parbox{0.8\textwidth}{%
                          \begin{itemize}[leftmargin=6ex, itemsep=-0.88ex, topsep=-0.88ex]
                              \item Rescinded due to schedule conflicts (primarily because of COVID-19). \\
                          \end{itemize}
                      }  
\\
                      & Selected for a \textbf{research internship} at HEPIA-Hesge, Geneva, Switzerland \hfill \textit{Mar 2020} \\
                      & to work on ``NavTrack: A portable obstacle tracker for the rehabilitation of spatial neglect''\\
                      & \parbox{0.8\textwidth}{%
                          \begin{itemize}[leftmargin=6ex, itemsep=-0.88ex, topsep=-0.88ex]
                              \item Received a grant of 4200CHF to conduct research under Prof. Florent Gluck, HEPIA.
                              \item Rescinded (Internship \& grant) due to COVID-19. \\
                          \end{itemize}
                      }  
\\


% Positions of responsibility


% References

\end{longtable}
\end{document}

