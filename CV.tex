% This CV is based on the:
% LaTeX Curriculum Vitae Template
%
% Copyright (C) 2004-2009 Jason Blevins <jrblevin@sdf.lonestar.org>
% http://jblevins.org/projects/cv-template/

\documentclass[letterpaper]{article}

\usepackage{hyperref}
\usepackage{geometry}
\usepackage{upgreek}
\usepackage{xcolor}
\usepackage{fontawesome}

% Name
\def\name{Russel Shawn Dsouza}

\def\footerlink{https://github.com/rshwndsz/resume/blob/master/CV.pdf}

% Metadata
\hypersetup{
  colorlinks = true,
  urlcolor = black,
  pdfauthor = {\name},
  pdfkeywords = {neuroscience, medical-imaging, signal-processing, deep-learning},
  pdftitle = {\name: Russel Dsouza's Curriculum Vitae},
  pdfsubject = {Russel Dsouza's Curriculum Vitae},
  pdfpagemode = UseNone
}

\geometry{
  body={6.5in, 10in},
  left=0.8in,
  top=0.8in
}

% Customize page headers
\pagestyle{myheadings}
\markright{\name}
\thispagestyle{empty}

% Custom section fonts
\usepackage{sectsty}
\sectionfont{\rmfamily\mdseries\Large}
\subsectionfont{\rmfamily\mdseries\itshape\large}

% Colors
\definecolor{linkedinblue}{HTML}{0077B5}
\definecolor{gmailred}{HTML}{D44638}
\definecolor{githubblack}{HTML}{333333}

% Don't indent paragraphs.
\setlength\parindent{0em}

% Make lists without bullets
\renewenvironment{itemize}{
  \begin{list}{}{
    \setlength{\leftmargin}{1.5em}
  }
}{
  \end{list}
}

% Custom commands
%% Smaller font size and lighter text color for dates
\newcommand{\smallGreyFont}[1]{\textcolor{black!80}{\small{#1}}}

\begin{document}

% Place name at left
{\huge \name}

\vspace{0.25in}

% Address
\begin{minipage}{0.5\linewidth}
  3rd year, Electronics \& Communications Engineering\\
  \href{http://www.nitk.ac.in/}{National Institute of Technology Karnataka} \\
  Surathkal, Mangalore \\
  Karnataka, India -- $575025$
\end{minipage}
% Contact Info
\hfill
\begin{minipage}{0.3\linewidth}
  \begin{tabular}{ll}
    \textcolor{gmailred}{\faEnvelopeO} \href{mailto:russel.171ec143@nitk.edu.in}{\tt russel.171ec143@nitk.edu.in} \\
    \textcolor{githubblack}{\faGithub} \href{https://www.github.com/rshwndsz}{\tt github.com/rshwndsz} \\
    \textcolor{linkedinblue}{\faLinkedin} \href{https://www.linkedin.com/in/rshwndsz}{\tt linkedin.com/in/rshwndsz}
  \end{tabular}
\end{minipage}


% Skills
\section*{Skills}
  \begin{itemize}
    \item \textbf{Research areas}\\
    Deep learning for computer vision\\
    Low power computer vision\\
    Medical image processing - Histopathology, MRI, CT
    \item \textbf{Programming languages}\\
    Python, MATLAB, C, JavaScript, C++, Verilog
    \item \textbf{Deep learning}\\
    PyTorch, Keras, scikit-learn
    \item \textbf{Image processing}\\
    OpenCV, scikit-image, PIL
    \item \textbf{Data mining \& analysis}\\
    Google BigQuery, SQL, requests, pandas
    \item \textbf{Web development}\\
    Django, ExpressJS, ReactJS, GatsbyJS, GraphQL
    \item \textbf{Hardware}\\
    Xilinx Vivado and Artix 7 fpga, ngSPICE, Raspberry Pi, Arduino
    \item \textbf{Tools}\\
    git, Docker, Linux,  \LaTeX
  \end{itemize}


% Projects
\section*{Notable Projects}
  \begin{itemize}
  \item
    \textbf{Detection of Urothelial Carcinoma from whole-slide images}
    \hfill{\smallGreyFont{Feb 2020 - Present}}\\
      Working on the design and development of an efficient automated algorithm for the  visual and semantically interpretable detection of Urothelial Carcinoma from whole slide images with an average dimension of 80,000 x 50,000 \\
      \texttt{\smallGreyFont{Technologies used: rshwndsz/histovision}}

    \item
    \textbf{Computational histopathology}
    \hfill{\smallGreyFont{Dec 2020 - Present}}\\
      Working on \texttt{rshwndsz/histovision}, a benchmark repository of state-of-the-art architectures and methods for segmentation and classification of nuclei in histopathology images.\\
      \texttt{\smallGreyFont{PyTorch, OpenCV, Nvidia-Dali}}

    \item
    \textbf{Cell nuclei segmentation}
    \hfill{\smallGreyFont{May 2019 - July 2019}}\\
      Implemented convolutional encoder-decoder architectures including U-Net, UNet with pyramid pooling and ResNets to perform the semantic segmentation of nuclei in H\&E stained histopathology images of kidney tissues.\\
      \texttt{\smallGreyFont{PyTorch, OpenCV}}

    \item
    \textbf{Brain tumour segmentation}
    \hfill{\smallGreyFont{Dec 2019}}\\
       Coded state-of-the-art semantic segmentation models and trained them on a part of the BRATS dataset to segment brain tumour and surrounding edema as a part of a 14-day workshop on Machine learning techniques in Neuroimaging.\\
       \texttt{\smallGreyFont{Keras, OpenCV}}

    \item
    \textbf{Classifying components of handwritten Bengali}
    \hfill{\smallGreyFont{Jan 2020 - Present}}\\
      Working on efficient, flexible models with low training times on single GPU systems for the Kaggle Bengali.AI Grapheme classification challenge.\\
      \texttt{\smallGreyFont{PyTorch, Nvidia-Dali}}

    \item
    \textbf{Detecting Ponzi schemes in Ethereum smart-contracts}
    \hfill{\smallGreyFont{Aug 2019 - Sep 2019}}\\
      Built a custom model using CNNs and stacked autoencoders and trained it on raw bytecode of Ethereum smart contracts mined from the blockchain to detect Ponzi schemes in Ethereum smart-contracts as a part of a 48-hour sprint to "apply deep learning on any part of the blockchain".\\
      \texttt{\smallGreyFont{PyTorch, torchtext, SQL, Google BigQuery, pandas}}

    \item
    \textbf{Predicting truth level of news articles}
    \hfill{\smallGreyFont{Jul 2019 - Aug 2019}}\\
      Built a classifier using a Bidirectional-LSTM, and trained it on the LIAR-PLUS dataset to classify news articles into 6 different categories based on their truth-level.\\
      \texttt{\smallGreyFont{PyTorch, torchtext}}

    \item
    \textbf{Spell checker}
    \hfill{\smallGreyFont{Oct 2018 - Nov 2018}}\\
      Built a command line application to correct spelling errors as a part of a course-project in Data Structures \& Algorithms.\\
      \texttt{\smallGreyFont{C, make}}

    \item
    \textbf{Space-time adaptive processing radar}
    \hfill{\smallGreyFont{Apr 2019 - May 2019}}\\
      Presented a report on space-time adaptive processing and simulated STAP in a radar as a part of a mini-project on Digital Signal Processing.\\
      \smallGreyFont{\texttt{MATLAB}, \LaTeX}

  \end{itemize}


% Experience
\section*{Experience}
  \begin{itemize}
    \item \textbf{Winter Research Intern}
    \hfill{\textcolor{black!80}{\small{Dec 2019 - Jan 2020}}}\\
    \smallGreyFont{Under Dr. Shyam Lal - NITK, India}\\
      Working on the design and development of an automated kidney \& colon cancer detection system from H\&E stained histopathology images.

    \item \textbf{Summer Research Intern}
    \hfill{\textcolor{black!80}{\small{May 2019 - Jul 2019}}}\\
    \smallGreyFont{Under Dr. Shyam Lal - NITK, India}\\
      Worked on reproducing state-of-the-art deep learning architectures for the semantic segmentation of H\&E stained histopathology images of kidney tissues.

    \item \textbf{Frontend Engineer}
    \hfill{\textcolor{black!80}{\small{Aug 2018 - Apr 2019}}}\\
    \smallGreyFont{IRIS-NITK, India}\\
      Worked on building the frontend for the official student management portal `IRIS' with more than five thousand daily active users including students, faculty, administrators and alumni.\\
      Mentored a freshman intern on frontend testing.

    \item \textbf{Part-time Python developer}
    \hfill{\textcolor{black!80}{\small{May 2018 - July 2018}}}\\
    \smallGreyFont{Pinnacle Media, Manipal, India}\\
      Worked on implementing real time face detection and recognition using open-cv, dlib and scikit-learn on a Raspberry Pi.
  \end{itemize}


% Publications
\section*{Publications}
  \begin{enumerate}
    \item Shyam Lal, Anirudh Kanfade, Kumar Alabhya, \textbf{Russel Dsouza}, Aman Kumar, Maneesh M, Gokul Perayil, Jyoti Kini \\
    \textbf{A Robust Method for Nuclei Segmentation of H\&E Stained Histopathology Images}\\
    IEEE 7th International Conference on Signal Processing and Integrated Networks (SPIN 2020), 27 - 28 February 2020, Amity University, Sec-125, Noida, Delhi-NCR, India -- \textit{\textbf{Accepted}}

    \item Shyam Lal, \textbf{Russel Dsouza}, Anirudh Kanfade, Kumar Alabhya, Aman Kumar, Maneesh M, Jyoti Kini
    \textbf{Deep Learning based Framework for Segmentation of H\&E Stained Histopathology Images of Kidney Tissues}\\
    IEEE Transactions of Medical Imaging, IEEE publisher.Indexed by SCI, Thomson ISI, Scopus (Elsevier), JCR (2018) Impact Factor: 2.770. -- \textit{Under preparation}
  \end{enumerate}


% Education
\section*{Education}
  \begin{itemize}
    \item National Institute of Technology Karnataka, India\hfill
    \smallGreyFont{2017-2021(expected)}
    \\
    \smallGreyFont{B.Tech in Electronics and Communications Engineering}
    \hfill{\smallGreyFont{CGPA: 8.7}}

    \item Little Rock Indian School, Karnataka, India\hfill
    \smallGreyFont{2004-2017}
    \\
    \smallGreyFont{K-12}
  \end{itemize}


% Academic courses
\section*{Course Work}
  \begin{itemize}
    \item  Digital signal processing in Python, Machine learning in neuroimaging, \\
    Digital system design in Verilog, Embedded system design,
    Microprocessors, Control sytems, VLSI design\\
    Numerical Analysis, Statistical Analysis, Data structures and algorithms, Digital electronics \& Computer architecture
  \end{itemize}


% Awards & Honors
\section*{Awards and Honors}
\begin{itemize}
  \item School topper in Math(99/100) and English(98/100) in Grade 12
  \item Top 1\%(CGPA 10.0) in India in Grade 10
\end{itemize}


% Interests
\section*{Interests}
  \begin{itemize}
    \item Computer vision, Neuroscience and Augmented Reality
  \end{itemize}


% Footer
\bigskip
\begin{center}
  \begin{footnotesize}
    Last updated: \today \\
    \href{\footerlink}{\texttt{\footerlink}}
  \end{footnotesize}
\end{center}

\end{document}
