% A simple CV
% === PREAMBLE ===
% Font size
\documentclass[letterpaper, 10pt, oneside]{article}
\usepackage[utf8]{inputenc}
\usepackage{setspace}
% For better tables
\usepackage{tabularx}
% For multi-page table
\usepackage{longtable}
% For more color names
\usepackage[dvipsnames]{xcolor}
% For Hyperlinks
\usepackage{hyperref}
% For better link colors
\definecolor{dark-purple}{HTML}{00034D}
\hypersetup{colorlinks  = true,
            linkcolor   = blue,
            filecolor   = magenta,
            urlcolor    = dark-purple, 
            pdftitle    = {Russel's CV},
            bookmarks   = true,
            pdfpagemode = FullScreen,
}
% For thicker horizontal rules
\usepackage{booktabs}
% For GitHub & LinkedIn icons
\usepackage{fontawesome}
% For margins
\usepackage[left=.64in, right=.64in, bottom=.64in, top=.64in]{geometry}
% For the thick underline - https://tex.stackexchange.com/a/443445
\usepackage{lipsum}
\newcommand\dunderline[3][-1pt]{{%
        \setbox0=\hbox{#3}
\ooalign{\copy0\cr\rule[\dimexpr#1-#2\relax]{\wd0}{#2}}}}
% For customization of list environment
\usepackage{enumitem}

% SETUP 
% Page numbers - {arabic}=Arabic numerals, {gobble}=no page numbers, {roman}=Roman numerals
\pagenumbering{arabic}


% === MY COMMANDS ===
% Section title - Uppercase with first letter slightly bigger
\newcommand{\stitle}[1]{\normalsize{\textsc{#1}}}
% Bold & Italic
\newcommand{\bdit}[1]{\textit{\textbf{#1}}}
% Use hyphens instead of bullets for lists
\def\labelitemi{--}
\begin{document}


% === ME ===
\noindent\dunderline[-2ex]{1pt}{\Large{\textbf{Russel Shawn Dsouza} \hspace{0.7\linewidth} }} \\
\normalsize


% === CV ===
\noindent \begin{longtable}{p{0.14\linewidth} p{0.64\linewidth} rp{0.15\linewidth}}
% Address & socials
\stitle{Contact}     & \href{https://nitk.ac.in}{National Institute of Technology Karnataka (NITK)} & \href{mailto:rshwndsz@gmail.com}{rshwndsz@gmail.com}                              \\
\stitle{Information} & NH66, Srinivasnagar, Surathkal, Mangalore                                    & \href{www.rshwndsz.github.io}{rshwndsz.github.io}                                 \\
                     & Karnataka, India 575025.  & \href{https://github.com/rshwndsz}{\faGithub} \href{https://linkedin.com/in/rshwndsz}{\faLinkedin} \textcolor{dark-purple}{rshwndsz} \\
\end{longtable}

\vspace{-1.2em}

\noindent \begin{longtable}{@{} p{0.14\linewidth} p{0.8\linewidth}}
% Research interests
% TODO General Interest
\stitle{Research}  & Computer vision, Neuroscience, Augmented Reality, Low power computing \\
\stitle{Interests} & \\
\\


% Education
\stitle{Education} & \textbf{National Institute of Technology Karnataka (NIT Karnataka)} \\
                   & Bachelor of Technology, Electronics and Communications Engineering \hfill \hspace{-3em} \textit{Jul 2017\ --\ May 2021}\\
\\


% Publications
\stitle{Publications} & Lal, S., \textbf{Dsouza, R.}, Maneesh, M., Kanfade, A., Kumar, A., Perayil, G., Alabhya, K., Chanchal, A.K. and Kini, J. \\
                      & \textit{``A Robust Method for Nuclei Segmentation of H\&E Stained Histopathology Images.''} \\
                      & 2020, 7th International Conference on Signal Processing and Integrated Networks (SPIN) (pp.~453--458)\@. IEEE\@.  \\
                      & \textcolor{dark-purple}{DOI\@: 10.1109/SPIN48934.2020.9070874} \\
\\


% Research experience
\stitle{Research}   & \bdit{Winter Research Intern, Deep learning lab, NIT Karnataka} \\
\stitle{Experience} & \bdit{Segmentation of nuclei in histopathology images of kidney, liver, bladder tissues} \\
                    & \textit{Mentored by Dr.\ Shyam Lal} \hfill \hspace{-3em} \textit{Dec 2019\ --\ Feb 2020} \\
                    & \parbox{0.8\textwidth}{
                        \begin{itemize}[leftmargin=*, itemsep=-0.88ex]
                            \item Worked on the segmentation and grading of kidney and liver cancer from histology images.
                            \item Worked on the detection of Urothelial Carcinoma from whole slide images with average dimensions of 80000$\times$50000 
                            \item Built an open-source repository benchmarking segmentation models on histopathology datasets 
                            \item Presented a report on various semantic and instance segmentation methods.
                        \end{itemize}
                    }
\\
                    & \bdit{Summer Research Intern, Deep learning lab, NIT Karnataka} \\
                    & \bdit{Segmentation of nuclei in histopathology images of kidney tissues} \\
                    & \textit{Mentored by Dr.\ Shyam Lal} \hfill \hspace{-3em} \textit{May 2019\ --\ Jul 2019} \\
                    & \parbox{0.8\textwidth}{
                        \begin{itemize}[leftmargin=*, itemsep=-0.88ex]
                            \item Worked on the efficient implementation of image processing algorithms on large datasets
                            \item Worked on reproducing the results of seminal papers in the field of automated histopathology.
                        \end{itemize}
                    }
\\


% Other Experience
\stitle{Work}       & \bdit{Frontend Developer} \\
\stitle{Experience} & \bdit{IRIS, NITK} \hfill \textit{Aug 2018\ --\ Apr 2019} \\
                    & \parbox{0.8\textwidth}{
                        \begin{itemize}[leftmargin=*, itemsep=-0.88ex]
                            \item Worked on building the frontend for the official student management portal for NITK --- `IRIS', which has more than five thousand
                                  daily active users including students, faculty, administrators, and alumni.
                            \item Mentored a freshman intern on frontend testing in JavaScript.
                        \end{itemize}
                    }
\\
                    & \bdit{Python Developer} \\
                    & \bdit{Pinnacle Media, Manipal} \hfill \textit{May 2018\ --\ Jun 2018} \\
                    & \parbox{0.8\textwidth}{
                        \begin{itemize}[leftmargin=*, itemsep=-0.88ex]
                            \item Built and deployed real-time face detection and recognition, using OpenCV, dlib, and scikit-learn, on a Raspberry Pi
                                  as a part of an `employee attendance' system.
                        \end{itemize}
                    }
\\


% Skills
\stitle{Skills} & \bdit{Languages}: C++, Python, MATLAB, Javascript, C, Rust, Verilog, ngSPICE \\
                & \bdit{Frameworks and packages}: Pytorch, Keras, OpenCV, Scikit-learn, Numerical Python \\
                & \bdit{Web Development}: ReactJS, ExpressJS, NodeJS, MongoDB, GraphQL \\
                & \bdit{Hardware}: Raspberry Pi, Arduino, Xilinx Artix 7 FPGA \\
                & \bdit{Natural languages}: English, Hindi, Kannada \\
\\


\newpage
% Notable projects
\stitle{Notable}  & \bdit{Identifying Melanoma in images of skin lesions} \hfill \textit{Jun 2020\ --\ Present} \\
\stitle{Projects} & \\
                  & \parbox{0.8\textwidth}{
                        \begin{itemize}[leftmargin=*, itemsep=-0.88ex, topsep=-0.1ex]
                            \item Working on building an ensemble network of multiple detection models to accurately detect skin cancer, 
                                specifically Melanoma, in images of skin lesions 
                                as a part of the SIIM-ISC Melanoma classification challenge on Kaggle. 
                        \end{itemize}
                    }
\\
\\
                  & \bdit{Satellite detection in images from low-cost telescopes} \hfill \textit{Jul 2020\ --\ Present} \\
                  & \parbox{0.8\textwidth}{
                        \begin{itemize}[leftmargin=*, itemsep=-0.88ex]
                            \item Working on the design and development of a model to detect orbiting objects in the geostationary ring,
                                from sequences of consecutive frames imaging unknown portions of the sky,
                                as a part of the 'spotGEO' competition by the European Space Agency (ESA).
                        \end{itemize}
                  }
\\
                  & \bdit{Image Denoising} \hfill \textit{July 2020} \\
                  & \parbox{0.8\textwidth}{
                        \begin{itemize}[leftmargin=*, itemsep=-0.88ex]
                            \item Reproduced a very deep persistent memory network to perform image restoration by removing noise and
                                predicting uncorrupted images.
                            \item The model was trained on images from the Berkeley Segmentation Dataset (BSDS300) and 
                                tested on a modified version of the CIFAR10 dataset.
                        \end{itemize}
                  }
\\
                  & \bdit{Brain Tumour Segmentation (BraTS)} \hfill \textit{Dec 2019} \\
                  & \parbox{0.8\textwidth}{
                        \begin{itemize}[leftmargin=*, itemsep=-0.88ex]
                            \item Reproduced state-of-the-art multi-class semantic segmentation models in Keras/TFv1 and
                                trained them on a part of the BraTS dataset to segment brain tumour and the surrounding edema from MRI images.
                        \end{itemize}
                    }  
\\
                  & \bdit{Detecting Ponzi schemes in smart contracts} \hfill \textit{Aug 2019\ --\ Sep 2019} \\
                  & \parbox{0.8\textwidth}{
                        \begin{itemize}[leftmargin=*, itemsep=-0.88ex]
                            \item Built a custom model to detect Ponzi smart contracts deployed on the Ethereum blockchain
                                using CNNs and stacked auto-encoders. 
                            \item The model was trained on the raw bytecode of Ethereum smart contracts mined from the Ethereum blockchain 
                                  using Google BigQuery, publicly available Solidity source code of popular smart contracts, 
                                  and a publicly available dataset of known Ponzi schemes.
                            \item The model was built in under 48h as a part of a deep-learning coding sprint.
                        \end{itemize}
                    }  
\\
                  & \bdit{Predicting truth level of news articles} \hfill \textit{Jul 2019\ --\ Aug 2019} \\
                  & \parbox{0.8\textwidth}{
                        \begin{itemize}[leftmargin=*, itemsep=-0.88ex]
                            \item Built a model to classify news articles into 6 different categories based on their truth level.
                            \item The model was trained on the LIAR-PLUS dataset containing news articles and
                                fact-checking justifications from trusted sources.
                        \end{itemize}
                    }  
\\
                  & \bdit{Space Time Adaptive Processing Radar} \hfill \textit{Apr 2019} \\
                  & \parbox{0.8\textwidth}{
                        \begin{itemize}[leftmargin=*, itemsep=-0.88ex]
                            \item This project involved presenting a report on the current state of STAP in Radar Signal Processing.
                            \item The report contained a MATLAB simulation of a radar implementing STAP.
                        \end{itemize}
                    }  
\\


% Relevant Coursework
\stitle{Relevant}   & Digital Signal Processing, Machine Learning for Neuroimaging \\
\stitle{Coursework} & Digital System Design, Statistical Analysis, Numerical Analysis \\
                    & Embedded System Design, Microprocessors, VLSI Design, Control Systems \\
                    & Data Structures \& Algorithms, Digital Electronics \& Computer Architecture \\
\\


% Positions of responsibility


% Achievements, awards & honours
% \stitle{Scholastic}   & -- Selected for an internship at HEPIA-Hesge, Geneva, Switzerland to work on  \\
% \stitle{Achievements} & \hspace{1em}``NavTrack: A portable obstacle tracker for the rehabilitation of spatial neglect''\\
                      % & -- Top scorer in Math and English in Grade 12 \\
%

% References


\end{longtable}
\end{document}

